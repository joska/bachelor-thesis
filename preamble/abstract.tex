
\newpage

\begin{abstract}
\setcounter{page}{2}
\thispagestyle{plain} 
\par\noindent At present, the Component Database for ATLAS Inner Detector at LHC is still in development and it’s missing several important parts and features. The goal of this bachelor thesis is to gather and analyse the requirements for the Production database for the ATLAS experiment Inner Tracker, propose the architecture and implement the API layer that meets the required demands.

\paragraph{The second part} of this thesis is dedicated to gathering of the data required to understand the problematic of the ATLAS experiment and cooperating institutions.

\paragraph{The third part} of the application describes the requirements for the API layer, based on the research

\paragraph{The fourth part} proposes the architecture of the API layer shows mockups for the example resources and describes the inner functionality of the API application.

\par Considering the nature of the database content, special care is given to the authorisation and user authentication to protect the data in the database.

\paragraph{The fifth part} of this thesis is the implementation of the functional, usable and deployable REST API layer for the database.

\paragraph{The sixth part} describes the automated test that should assure the functionality of the database even when its functionality is extended in the future.

\paragraph{The seventh part} is the implementation of the simple web interface, which would use the API to provide access to stored data.

\paragraph{Keywords:} CERN, LHC, ATLAS, ITk, SCT, Component database, Production database, API, REST

\end{abstract}

\pagebreak